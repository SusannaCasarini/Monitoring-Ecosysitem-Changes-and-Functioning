\documentclass{beamer}
\usetheme{Dresden}
\usecolortheme{dolphin}
\usepackage{hyperref}

\title{Spatiotemporal analysis of Lake Manchar}
\subtitle{Using Landsat 8 and 9 satellite images}

\author{Susanna Casarini}
\date{\today}

\begin{document}

\begin{frame}
  \titlepage
  \centering
  \href{https://github.com/SusannaCasarini}{https://github.com/SusannaCasarini}  

\end{frame}

\section{Introduction}
\begin{frame}
  \frametitle{\small Introduction}
  
  \begin{itemize}
      \item \small Lake Manchar is the largest lake in Pakistan.
      \item \small Images obtained from the Landsat 8 and Landsat 9 satellites.
      \item \small In the summer of 2022, the lake overflowed, causing severe damage.
  \end{itemize}

    \hspace{1cm}
  
  \begin{minipage}{0.48\textwidth}
    \centering
    \includegraphics[width=0.6\linewidth]{0.1.png}
    \caption{}
  \end{minipage}
  \hfill
  \begin{minipage}{0.48\textwidth}
    \centering
    \includegraphics[width=0.6\linewidth]{0.png}
  \end{minipage}
  
 \end{frame}  
 
\begin{frame}
    \small Analysis:
    \begin{itemize}
        \item \small \textbf{First analysis}: Analysis of Lake Manchar in September from 2021 to 2024 - Source USGS - \href{https://earthexplorer.usgs.gov/}{https://earthexplorer.usgs.gov/}
        \item \small \textbf{Second analysis}: Analysis of Lake Manchar seasonality before and after the flood - Source USGS - \href{https://earthexplorer.usgs.gov/}{https://earthexplorer.usgs.gov/}
        \item \small \textbf{Third analysis}: Analysis of Lake Manchar NASA images before and after the flood - Source NASA Earth Observatory - \href{https://earthobservatory.nasa.gov/images/150306/lake-manchar-is-overflowing}{https://earthobservatory.nasa.gov/images/150306/lake-manchar-is-overflowing}
    \end{itemize}
\end{frame}

\section{First analysis}

\begin{frame}
  \frametitle{\small Analysis of Lake Manchar in September from 2021 to 2024} 
  \small \textbf{First analysis}:
    \begin{itemize}
        \item \small Temporal analysis of September from 2021 to 2024.
        \item \small Use of bands B2, B3, B4, and B5.
        \item \small Landsat 8 and 9 satellite.
        \item \small The images have been cropped and enhanced for better distinguishability.
    \end{itemize}
 
\end{frame}

\begin{frame}
        \begin{figure}
        \centering
        \includegraphics[width=0.5\linewidth]{1.png}
        \label{fig:enter-label}
        \caption{Natural color - Manchar Lake September 2021}
    \end{figure}

    \begin{figure}
        \centering
        \includegraphics[width=0.9\linewidth]{Extent.png}
        \label{fig:enter-label}
    \end{figure}

\begin{figure}
    \centering
    \includegraphics[width=0.7\linewidth]{Natural.png}
    \label{fig:enter-label}
\end{figure}
    
\end{frame}

\begin{frame}
  \frametitle{\small Analysis of Lake Manchar in September from 2021 to 2024} 
    \centering
    
    \begin{figure}
        \includegraphics[width=0.4\linewidth]{2.png}
        \label{fig:enter-label}
        \caption{Natural color images, using bands B2, B3, and B4}
    \end{figure}
  
\end{frame}

\begin{frame}
  \frametitle{\small Analysis of Lake Manchar in September from 2021 to 2024} 

    \begin{figure}
        \centering
        \includegraphics[width=0.4\linewidth]{3.png}
        \label{fig:enter-label}
        \caption{False color images, using red for infrared (NIR)}
    \end{figure}
  
\end{frame}

%\begin{frame}
%  \frametitle{\small Analysis of Lake Manchar in September from 2021 to 2024} 

%    Immagini a infrarosso
%    \begin{figure}
%        \centering
%        \includegraphics[width=0.5\linewidth]{4.png}
%        \caption{Enter Caption}
%        \label{fig:enter-label}
%    \end{figure}
  
% \end{frame}

\begin{frame}
  \frametitle{\small Analysis of Lake Manchar in September from 2021 to 2024} 

    \begin{figure}
        \centering
        \includegraphics[width=0.6\linewidth]{5.png}
        \label{fig:enter-label}
        \caption{Years B5 correlation - Low correlation for 2022}
    \end{figure}
  
\end{frame}

\begin{frame}
  \frametitle{\small Analysis of Lake Manchar in September from 2021 to 2024} 

   \begin{minipage}{0.48\textwidth}
    \centering
    \includegraphics[width=0.95\linewidth]{6.png}
    \caption{Water difference 2022-2023}
  \end{minipage}
  \hfill
  \begin{minipage}{0.48\textwidth}
    \centering
    \includegraphics[width=0.85\linewidth]{7.png}
    \caption{RGB in years 2021-2023}
  \end{minipage}
  
\end{frame}  

\begin{frame}
  \frametitle{\small Analysis of Lake Manchar in September from 2021 to 2024} 

    \begin{itemize}
        \item \small Classification with im.classify
        \item \small Three classes: water, vegetation, desert
    \end{itemize}
    
    \begin{figure}
        \centering
        \includegraphics[width=0.6\linewidth]{8.png}
        \label{fig:enter-label}
        \caption{Classification of 2021 (left) and 2022 (right)}
    \end{figure}

    \begin{figure}
        \centering
        \includegraphics[width=0.5\linewidth]{classification.png}
        \label{fig:enter-label}
    \end{figure}
  
\end{frame}

\begin{frame}
  \frametitle{\small Analysis of Lake Manchar in September from 2021 to 2024} 

  \begin{columns} 
    \begin{column}{0.45\textwidth}  
      \centering
      \begin{tabular}{|c|c|}
      \hline
      \multicolumn{2}{|c|}{Water percentage} \\ 
      \hline
      \textbf{Year} & \textbf{Water} \\  
      \hline
      2021 & 4.64 \\
      \hline
      2022 & 31.71 \\
      \hline
      2023 & 6.93 \\
      \hline
      2024 & 9.35 \\
      \hline
      \end{tabular}
    \end{column}

    \begin{column}{0.45\textwidth}  
      \centering
      \includegraphics[width=\linewidth]{10.png}
      \label{fig:enter-label}
    \end{column}
  \end{columns}  

\begin{figure}
    \centering
    \includegraphics[width=0.5\linewidth]{ClassificationPerc.png}
    \label{fig:enter-label}
\end{figure}

\end{frame}

\begin{frame}
  \frametitle{\small Analysis of Lake Manchar in September from 2021 to 2024} 

    \centering
    \small \begin{equation}
NDVI = \frac{NIR - RED}{NIR + RED}
\end{equation}
    
        \begin{figure}
        \centering
        \includegraphics[width=0.8\linewidth]{11.png}
        \caption{NDVI index in 2021 (left) and 2022 (right)}
        \label{fig:enter-label}
    \end{figure}
  
\end{frame}

\begin{frame}
  \frametitle{\small Analysis of Lake Manchar in September from 2021 to 2024} 

    \small \textbf{Conclusions of the first analysis}:
    \begin{itemize}
        \item \small Comparing the satellite images of the lake from 2021 to 2024, a clear difference in the amount of water present in the images after the flood can be observed.
        \item \small The infrared band allowed for better distinction between vegetation areas and lake areas.
        \item \small The classification highlighted the differences between the years.
    \end{itemize} 
  
\end{frame}

\section{Second analysis}

\begin{frame}
  \frametitle{\small Analysis of Lake Manchar seasonality before and after the flood} 

  \small \textbf{Second analysis}:
  \begin{itemize}
      \item \small Comparison of the images in the 4 seasons of the year preceding the flood and the year of the flood.
      \item \small Classification and comparison of the percentages of water, vegetation, and desert in the different seasons.
      \item \small NIR band.
  \end{itemize}
  
\end{frame}

\begin{frame}

  \begin{figure}
      \centering
      \includegraphics[width=\linewidth]{12.png}
      \label{fig:enter-label}
      \caption{NIR band for each season - September 2021 - June 2023}
  \end{figure}    
  
\end{frame}
    
\begin{frame}
  \frametitle{\small Analysis of Lake Manchar seasonality before and after the flood} 

\begin{figure}
    \centering
    \includegraphics[width=0.8\linewidth]{13.png}
    \label{fig:enter-label}
    \caption{Clustering and comparison between cluster percentages}
\end{figure}
  
\end{frame}
  
\begin{frame}
      \frametitle{\small Analysis of Lake Manchar seasonality before and after the flood} 

    \small \textbf{Conclusions of the second analysis}:
    \begin{itemize}
        \item \small The seasonal differences are not as significant compared to the consequences of the flood. 
        \item \small In the winter following the flood, the amount of water is higher than usual. 
        \item \small From spring onwards, the levels return to normal.
    \end{itemize}
\end{frame}

\section{Third analysis}

\begin{frame}
  \frametitle{\small Analysis of Lake Manchar NASA images before and after the flood} 
  
    \small \textbf{Third analysis}:
    \begin{itemize}
        \item \small Use of NASA Landsat 8 and 9 images with bands B2, B3, B4 of June 2022 and September 2022.
        \item \small PCA analysis of June and September: PC1 variability and differences.
        \item \small Density analysis.
    \end{itemize}  
\end{frame}

\begin{frame}
    \begin{figure}
        \centering
        \includegraphics[width=\linewidth]{14.png}
        \label{fig:enter-label}
        \caption{Lake Manchar - June 2022 and September 2022 - NASA source}
    \end{figure}    
\end{frame}

\begin{frame}
\small\centering Code to perform PCA and study variability of PC1
        \begin{figure}
        \centering
        \includegraphics[width=0.3\linewidth]{PCA.png}
        \label{fig:enter-label}
    \end{figure}

    \begin{figure}
        \centering
        \includegraphics[width=0.5\linewidth]{pc1sd.png}
        \label{fig:enter-label}
    \end{figure}
\end{frame}

\begin{frame}
  \frametitle{\small Analysis of Lake Manchar NASA images before and after the flood} 

    \begin{figure}
        \centering
        \includegraphics[width=0.6\linewidth]{16.png}
        \caption{First principal component of September 2022}
        \label{fig:enter-label}
    \end{figure}

\end{frame}

\begin{frame}
  \frametitle{\small Analysis of Lake Manchar NASA images before and after the flood} 

\begin{figure}
    \centering
    \includegraphics[width=0.6\linewidth]{18.png}
    \label{fig:enter-label}
    \caption{September variability of the first component}
\end{figure}
  

\end{frame}

\begin{frame}
  \frametitle{\small Analysis of Lake Manchar NASA images before and after the flood} 

   \begin{minipage}{0.48\textwidth}
    \centering
    \includegraphics[width=\linewidth]{19.png}
    \caption{June PC1}
  \end{minipage}
  \hfill
  \begin{minipage}{0.48\textwidth}
    \centering
    \includegraphics[width=\linewidth]{20.png}
    \caption{September/June differences}
  \end{minipage}


\end{frame}

\begin{frame}
  \frametitle{\small Analysis of Lake Manchar NASA images before and after the flood} 

  \begin{figure}
      \centering
      \includegraphics[width=0.9\linewidth]{22.png}
      \label{fig:enter-label}
      \caption{B2, B3 and B4 band differences between September and June}
  \end{figure}

\end{frame}

%\begin{frame}
%  \frametitle{\small Analysis of Lake Manchar NASA images before and after the flood} 
%\begin{figure}
%    \centering
%    \includegraphics[width=0.5\linewidth]{24.png}
%    \caption{Enter Caption}
%    \label{fig:enter-label}
%\end{figure}
%\end{frame}

\begin{frame}
  \frametitle{\small Analysis of Lake Manchar NASA images before and after the flood} 

  \centering\small Bands density in June and September
  \vspace{1cm}

   \begin{minipage}{0.48\textwidth}
    \centering
    \includegraphics[width=\linewidth]{25.png}
  \end{minipage}
  \hfill
  \begin{minipage}{0.48\textwidth}
    \centering
    \includegraphics[width=\linewidth]{26.png}
    \caption{}
  \end{minipage}
\begin{figure}
    \centering
    \includegraphics[width=0.7\linewidth]{density.png}
    \label{fig:enter-label}
\end{figure}

\end{frame}

\section{Conclusions}
\begin{frame}
  \frametitle{\small Conclusions}
  
    \small \textbf{Conclusions}:
    \begin{itemize}
        \item \small The amount of water in the months following the flood is much higher compared to the quantities in previous years. 
        \item \small Seasonality usually does not significantly affect the amount of water and vegetation, compared to the increase in water caused by the flood. 
        \item \small It is observed that within a few months, the water levels returned to normal, although in December 2022 a higher amount of water is seen compared to the previous winter. 
        \item \small The NIR band proved useful for distinguishing water and vegetation, but it was also possible to perform an accurate classification and study the images using bands B2, B3, and B4.
    \end{itemize}
  
\end{frame}

\begin{frame}
 \centering\huge Thank you!    
\end{frame}

\end{document}
